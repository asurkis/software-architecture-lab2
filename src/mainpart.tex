\section*{Выполнение}

\subsection*{<<Одиночка>> (GoF Singleton)}
Класс, у которого существует только один экземпляр с глобальным доступом.
Иногда синглтоном также называют класс, у которого ограниченное заранее заданных экземпляров.
Примеры применения:
\begin{enumerate}
    \item Хранение ресурсов: текстур, звуковых эффектов, моделей.
    Эти данные обычно неизменяемые и массивные, поэтому их крайне желательно не дублировать.

    Ограничение: ресурсы обязательно должны быть неизменяемыми,
    поэтому при наличии, например, динамически генерируемых 3D-моделей разрушаемых объектов,
    их хранилище нужно организовывать отдельно от загружаемых с диска,
    из-за чего доступ получается не унифицированным.

    \item Доступ к глобальному состоянию в силу ограничений используемых API,
    например, OpenGL не поддерживает множественные контексты в разных потоках,
    поэтому желательно организовать непосредственно код и контекст отрисовки как синглтон,
    обеспечивающий синхронизацию.

    Ограничение: фактически синглтон на стороне разработчика API (глобальное состояние)
    ограничивает использование этого API в многопоточных программах.

    \item Логирование --- обычно логи должны записываться в один и тот же выходной поток на протяжении всего
    времени жизни приложения.
    Поэтому, чтобы не передавать один и тот же экземпляр класса для логирования всем частям программы,
    можно сделать его синглтоном.

    Ограничение: если в процессе разработки изменится логика,
    и потребуется логи некоторой части программы перенести в другой выходной поток,
    например, если изначально сервер создавал новый процесс для каждой сессии,
    для улучшения производительности нужно, чтобы один процесс обрабатывал несколько сессий,
    то для сохранения логики ,,одна игра --- один лог'' нужно выковыривать глобальный для процесса логгер.
\end{enumerate}

\subsection*{<<Команда>> (GoF Command)}
Представление запроса в виде объекта (структуры данных) для передачи исполнителю.
Примеры применения:
\begin{enumerate}
    \item Передача ввода пользователя на сервер для обработки игровой логики.
    Клиент может заранее обработать такой запрос (например, преобразовать клик мышкой по юниту X
    на экране из координат мыши и позиции камеры в команду <<выделение юнита X>>),
    и может предполагать результат его выполнения при отрисовке для компенсации сетевой задержки,
    но <<эталонная>> игровая логика и проверка корректности операций остается на сервере.

    Ограничение: обработка команды всегда производится со значительной задержкой
    по сравнению с непосредственным вызовом функции, поэтому, например,
    для одиночных игр не имеет смысла кодировать ввод пользователя и ставить его в очередь на обработку.

    \item Контекст для сложных операций, например, отрисовки.
    Обычно гораздо проще создать один контекст, менять в нем какие-то параметры и
    повторно передавать в команды отрисовки, чем каждый раз явно указывать все параметры функции.
    Такой контекст используется почти во всех API и библиотеках, связанных с графикой
    (простейшие примеры --- Java~AWT, Swing, HTML5~Canvas, WinAPI, SDL, SFML,
    более сложный, но и более правильный --- командные буферы в Vulkan~API)
    и в целом может считаться командой, так как заранее сохраняет параметры функции в виде объекта,
    который можно переиспользовать.

    Ограничения: для простых операций, принимающих мало параметров, не нужен контекст,
    он лишь усложняет вызов функции.

    \item Сохранение истории действий и машиночитаемое логирование,
    например, для воспроизведения ошибок.

    Ограничение: машиночитаемые логи могут занимать сильно больше места на диске,
    при этом быть не более полезными, чем краткие описания для человека,
    при этом для них нужно использовать специализированный интерпретатор.
\end{enumerate}

\subsection*{Контроллер (GRASP)}
Концепция разделения интерфейса представления и модели данных промежуточным слоем --- контроллером,
основная задача которого --- диспетчеризация запросов и ответов
между интерфейсом (представлением) и реализацией (моделью).
Примеры применения:
\begin{enumerate}
    \item Model-View-Controller в разработке веб-приложений.
    Обычно и модель данных, и пользовательский интерфейс являются достаточно сложными,
    причем не взаимосвязанными напрямую.
    Контроллер перенаправляет запросы интерфейса по разным функциям модели,
    и упаковывает ответ от модели в представление для интерфейса.

    Ограничения: в некоторых случаях модель и представление все же являются сильно похожими,
    или очень простыми.
    Нет смысла разворачивать полноценный Spring для сайта,
    основная задача которого --- постранично показывать веб-комикс,
    для этого достаточно PHP,
    или даже no-code инструменты.
    В такой ситуации контроллер был бы заделом на неопределенное будущее,
    при этом оставаясь лишним кодом, который по сути ничего не делает.

    \item Парсинг и проверка корректности ввода --- если пользовательский запрос (в том числе в веб-приложении)
    некорректен, в большинстве случаев это можно определить еще на уровне контроллера:
    текст в поле для числа, число вне заданного диапазона, адрес электронной почты без @
    --- многие подобные ошибки можно поймать еще на этапе преобразования значений
    из компонент пользовательского интерфейса в команды (паттерн <<Command>>, см. выше),
    и вернуть в представление (сообщить пользователю), не передавая в модель.
    Это позволяет упростить реализацию модели, вынося из нее проверку корректности запроса в контроллер.

    Ограничения: существуют ошибки, где ограничение на тип данных не дает достаточных проверок корректности.
    Например, в запросе <<передать юниту X команду следовать в точку Y>>,
    существование юнита X нельзя определить на уровне контроллера,
    и проверить это можно только уже зная список юнитов.
    Поэтому проверки бизнес-логики все же придется оставить в реализации модели.
\end{enumerate}

\subsection*{Полиморфизм (GRASP)}
Концепция реализации одного интерфейса разными сущностями.
Примеры использования:
\begin{enumerate}
    \item Файловая система UNIX-подобных операционных систем.
    Здесь файл --- абстракция не над областью на диске, а над потоком данных, который можно читать и/или записывать.
    Например, файл \verb|/dev/urandom| не представлен ни на диске, ни в оперативной памяти,
    а является интерфейсом для системного генератора псевдослучайных чисел.
    Так системные вызовы \verb|read| и \verb|write| являются полиморфными,
    и для разных файлов производят разные операции.

    Ограничения: использование концепции файла как абстрактного потока не включает возможность
    чтения/записи по произвольному адресу файлов на твердотельных накопителях или
    в виртуальной файловой системе в оперативной памяти, где эта операция возможна.
    Для этого необходимо использовать другую абстракцию --- отображение файла в память.
    Таким образом, использование общего простого интерфейса не позволяет эффективно использовать
    возможности отдельных реализаций.

    \item Отрисовка в OpenGL, Direct3D, Vulkan.
    В этих API существует концепция шейдеров --- программ, выполняющихся на видеокарте над каким-то фрагментом данных:
    вершинный шейдер обрабатывает каждую вершину при отрисовке, пиксельный --- каждый пиксель после растеризации.
    Есть и другие виды шейдеров, но этих пока достаточно.
    Шейдерная программа задается набором шейдеров, необходимых для какой-то операции (отрисовка, вычисления).
    После создания шейдерной программы API для ее использования зависит только от типа операции,
    но не от самих шейдеров.
    Так, запрос отрисовки 6 вершин как списка из двух треугольников в главном цикле
    не будет зависеть от того, монотонный или градиентный цвет у этих треугольников,
    используют ли они текстуры и освещение и т.п., только количество вершин и вид растеризации
    (преобразования списка вершин в пространстве OpenGL/Direct3D/Vulkan в набор пикселей).

    Ограничения: подобный подход увеличивает гибкость за счет простоты базовых случаев.
    Простейшая программа на OpenGL~3+ значительно больше аналогичной на OpenGL~1--2,
    и хотя именно отрисовка нескольких треугольников вряд ли является типовой задачей,
    разница в сложности уже между OpenGL~4 и Vulkan
    (где еще более тонкая настройка и больше полиморфизма)
    показывает, что существует компромисс между гибкостью и простотой.
\end{enumerate}
